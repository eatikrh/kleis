\documentclass[12pt]{article}
\usepackage{amsmath, amssymb, amsthm, mathrsfs, stmaryrd}
\usepackage{geometry}
\geometry{margin=1in}

\begin{document}

\section*{Appendix A. Bourbaki–Style Foundations of Kleis}

\subsection*{A.0. Introduction}

The present appendix establishes, in a form adhering to the structural method,
the foundational principles upon which the Kleis language is constructed.
Kleis is intended to serve as a formal system for the description of
mathematical structures, their relations, and the operations defined upon them.
Rather than adopting a computational viewpoint, we adopt here the classical
framework of \emph{structures} as introduced in the \emph{Éléments de
mathématique}.

Throughout, we assume the existence of a background universe $\mathscr{U}$
sufficiently large to contain all sets considered. No distinction is made
between sets and classes except where explicitly noted. All constructions are
understood to take place within $\mathscr{U}$.

\bigskip

%%%%%%%%%%%%%%%%%%%%%%%%%%%%%%%%%%%%%%%%%%%%%%%%%%%%%%%%%%%%%%
\subsection*{A.1. Species of Structures}

A \emph{species of structure} in the sense of Bourbaki is determined by:
\begin{itemize}
    \item[(i)] a finite family of underlying sets,
    \item[(ii)] a finite family of operations of finite arity,
    \item[(iii)] a finite family of relations of finite arity,
    \item[(iv)] a finite family of axioms expressed as identities or implications.
\end{itemize}

In Kleis, such a species corresponds precisely to a declaration of the form:

\begin{verbatim}
structure S(X1, ..., Xn) { ... }
\end{verbatim}

The sets $X_1, \dots, X_n$ constitute the underlying sets of the structure;
the operations and relations correspond to the \texttt{operation}
and \texttt{element} declarations; the axioms correspond to the \texttt{axiom}
declarations.

Thus, a Kleis structure is a \emph{formal species of structure} in
Bourbaki’s sense.

\bigskip

%%%%%%%%%%%%%%%%%%%%%%%%%%%%%%%%%%%%%%%%%%%%%%%%%%%%%%%%%%%%%%
\subsection*{A.2. Morphisms of Species and the \texttt{extends} Relation}

Given two species of structures $\mathcal{S}$ and $\mathcal{T}$,
a \emph{morphism of species} is any mapping which respects:
\begin{itemize}
    \item the number and positions of underlying sets,
    \item the signature of operations and relations,
    \item the axioms imposed.
\end{itemize}

The Kleis construct:

\begin{verbatim}
structure T(...) extends S(...)
\end{verbatim}

represents precisely such a morphism of species in which:
\begin{itemize}
    \item each operation of $\mathcal{S}$ is preserved in $\mathcal{T}$,
    \item each axiom of $\mathcal{S}$ holds in $\mathcal{T}$,
    \item additional operations and axioms may be added.
\end{itemize}

Thus, \texttt{extends} corresponds to the inclusion of one species of structure
into another.

\bigskip

%%%%%%%%%%%%%%%%%%%%%%%%%%%%%%%%%%%%%%%%%%%%%%%%%%%%%%%%%%%%%%
\subsection*{A.3. Structures Dependent on Other Structures and the \texttt{over} Clause}

In Bourbaki’s formalism, certain structures are defined relative to a
previously chosen structure.
For example, a vector space is defined only after a field has
been designated.

Kleis formalizes this by the construct:

\begin{verbatim}
structure V(X) over Field(F) { ... }
\end{verbatim}

This expresses that $V$ is a species whose definition presupposes the choice
of a structure of type \texttt{Field}.
Formally, this corresponds to the notion of a \emph{parametrized species of
structure}, or, in categorical language, to a \emph{fibred} family of
structures indexed by the models of the parameter species.

\bigskip

%%%%%%%%%%%%%%%%%%%%%%%%%%%%%%%%%%%%%%%%%%%%%%%%%%%%%%%%%%%%%%
\subsection*{A.4. Realizations of Structures and the \texttt{implements} Clause}

A \emph{structure of species} $\mathcal{S}$ in the sense of Bourbaki is given
by:
\begin{itemize}
    \item a family of sets,
    \item interpretations of each operation symbol as a genuine function,
    \item interpretations of relations as subsets,
    \item verification that all axioms hold.
\end{itemize}

In Kleis:

\begin{verbatim}
implements Field(ℝ) { ... }
\end{verbatim}

declares that the species \texttt{Field} is realized on the set $\mathbb{R}$.
The operations and elements given in the implementation constitute the
interpretation of the signature, and the axioms are asserted to hold in this
interpretation.

Thus \texttt{implements} corresponds exactly to the
Bourbaki notion of a \emph{structure of a given species}.

\bigskip

%%%%%%%%%%%%%%%%%%%%%%%%%%%%%%%%%%%%%%%%%%%%%%%%%%%%%%%%%%%%%%
\subsection*{A.5. Conditions on Realizations and the \texttt{where} Clause}

Many mathematical constructions impose side conditions on underlying sets or
operations (e.g., commutativity, order-completeness).
In Bourbaki’s methodology, these appear as \emph{subspecies of structures}.

The Kleis construct:

\begin{verbatim}
implements R(T) where C(T)
\end{verbatim}

specifies that the realization of $R$ on $T$ is valid only in cases where
$T$ satisfies an auxiliary species $C$.

Thus, \texttt{where} produces a \emph{subspecies} of the original species,
in the Bourbaki sense.

\bigskip

%%%%%%%%%%%%%%%%%%%%%%%%%%%%%%%%%%%%%%%%%%%%%%%%%%%%%%%%%%%%%%
\subsection*{A.6. Nested Structures and Composite Species}

A Bourbaki structure may itself contain other structures whose underlying
set is identical.
For example, a ring consists of an additive commutative group and a
multiplicative monoid defined on the same set.

In Kleis:

\begin{verbatim}
structure Ring(R) {
    structure additive : AbelianGroup(R)
    structure multiplicative : Monoid(R)
}
\end{verbatim}

This corresponds to a composite species of structure in Bourbaki’s
terminology: the ring structure is obtained by amalgamating two subspecies
defined on the same underlying set and adding compatibility axioms.

Thus, nested Kleis structures represent the Bourbaki notion of a structure
\emph{with internal substructures of species}.

\bigskip

%%%%%%%%%%%%%%%%%%%%%%%%%%%%%%%%%%%%%%%%%%%%%%%%%%%%%%%%%%%%%%
\subsection*{A.7. Algebraic Data Types and Inductive Species}

An inductive Kleis data type:

\begin{verbatim}
data Option(T) = None | Some(T)
\end{verbatim}

corresponds to the initial model of a finitary algebraic species.
In Bourbaki's terminology, this is a \emph{free structure} on the given
signature.

Given a functor:
\[
F(X) = 1 + T \times X,
\]
the type defined above is the initial object of the category of
$F$-algebras.

\bigskip

%%%%%%%%%%%%%%%%%%%%%%%%%%%%%%%%%%%%%%%%%%%%%%%%%%%%%%%%%%%%%%
\subsection*{A.8. Axioms as Structural Relations}

A Kleis axiom:

\begin{verbatim}
axiom associativity:
    ∀(x y z : M). (x • y) • z = x • (y • z)
\end{verbatim}

corresponds to an identity constraint on the operations of the species.
In Bourbaki’s setting, these axioms complete the specification of the
structure by imposing structural relations on the operations.

\bigskip

%%%%%%%%%%%%%%%%%%%%%%%%%%%%%%%%%%%%%%%%%%%%%%%%%%%%%%%%%%%%%%
\subsection*{A.9. Functoriality and Polymorphism}

Many Kleis expressions, particularly polymorphic type constructors, correspond
to functors between categories of structures.

For a type operator such as:

\begin{verbatim}
Matrix(m,n,T)
\end{verbatim}

we obtain a mapping:
\[
\mathsf{Matrix}_{m,n} : \mathscr{U} \to \mathscr{U},
\]
which, when restricted to appropriate categories of structures, becomes a
covariant functor.

Polymorphic Kleis functions correspond to families of functions natural
with respect to such functorial actions.

\bigskip

%%%%%%%%%%%%%%%%%%%%%%%%%%%%%%%%%%%%%%%%%%%%%%%%%%%%%%%%%%%%%%
\subsection*{A.10. Summary}

In this appendix, we have identified the principal constructs of the Kleis
language with the corresponding notions in Bourbaki’s theory of structures:

\begin{center}
\begin{tabular}{|c|c|}
\hline
\textbf{Kleis Construct} & \textbf{Bourbaki Concept} \\
\hline
\texttt{structure} & species of structure \\
\texttt{extends} & morphism of species / species inclusion \\
\texttt{over} & parametrized species \\
\texttt{implements} & structure of a given species \\
\texttt{where} & subspecies via conditions \\
nested structures & composite species / associated structures \\
algebraic data types & free or inductive structures \\
polymorphism & functorial dependence \\
\hline
\end{tabular}
\end{center}

Thus, Kleis may be regarded as a formal, executable language for defining
and manipulating Bourbaki-style mathematical structures.

\end{document}