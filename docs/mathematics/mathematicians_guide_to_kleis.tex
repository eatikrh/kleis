\documentclass[12pt]{article}
\usepackage{amsmath, amssymb, amsthm, stmaryrd, mathrsfs}
\usepackage{bm}
\usepackage{geometry}
\geometry{margin=1in}

\title{\textbf{A Mathematician’s Guide to Kleis}}
\author{Kleis Language Project}
\date{}

\begin{document}

\maketitle

\section*{Introduction}

Kleis is a formally structured, algebra-oriented language designed to express
mathematical objects and their axioms with precision.
Unlike programming languages that treat structures as implementation artifacts,
Kleis models mathematical practice directly:

\begin{itemize}
    \item Abstract structures (groups, rings, fields)
    \item Parameterized structures (vector spaces over a field)
    \item Substructures (additive and multiplicative components of a ring)
    \item Explicit implementations of axioms for concrete models (e.g.\ $\mathbb{R}$ is a field)
\end{itemize}

This document introduces the central structural features of Kleis:
\texttt{extends}, \texttt{over}, \texttt{implements}, \texttt{where},
and nested structures.

Each feature corresponds to a familiar concept in algebra or category theory.

%%%%%%%%%%%%%%%%%%%%%%%%%%%%%%%%%%%%%%%%%%%%%%%%%%%%%%%%%%%%%%%
\section{The \texttt{extends} Construct: Inheritance of Structure}

When mathematicians define a group, they implicitly note that every group
is also a monoid.
Formally:

\[
(G, \cdot, e, ()^{-1}) \text{ is a group } \implies
(G, \cdot, e) \text{ is a monoid}.
\]

Kleis expresses this relationship directly:

\begin{verbatim}
structure Group(G) extends Monoid(G) {
    operation inv : G → G
    axiom left_inverse:
        ∀(x : G). inv(x) • x = e
}
\end{verbatim}

Thus:

\begin{itemize}
    \item all operations of a monoid are \emph{inherited},
    \item all axioms of a monoid apply automatically to groups,
    \item new axioms extend the theory.
\end{itemize}

This mirrors the extension of algebraic theories in universal algebra.

%%%%%%%%%%%%%%%%%%%%%%%%%%%%%%%%%%%%%%%%%%%%%%%%%%%%%%%%%%%%%%%
\section{The \texttt{over} Construct: Parameterized Structures}

Many mathematical structures require a second structure as part of their
definition.
For example, a vector space is defined:

\[
(V, +, \cdot) \quad \text{over a field } F.
\]

In Kleis:

\begin{verbatim}
structure VectorSpace(V) over Field(F) {
    operation (+) : V × V → V
    operation (·) : F × V → V
}
\end{verbatim}

This expresses that:

\begin{itemize}
    \item the structure \texttt{VectorSpace(V)} is not meaningful alone,
    \item it depends on the existence of a field \(F\),
    \item its axioms relate \(V\) and \(F\).
\end{itemize}

This corresponds exactly to parameterized algebraic structures (modules,
vector spaces, algebras) in standard mathematical literature.

%%%%%%%%%%%%%%%%%%%%%%%%%%%%%%%%%%%%%%%%%%%%%%%%%%%%%%%%%%%%%%%
\section{The \texttt{implements} Construct: Concrete Mathematical Models}

In mathematics, once an axiomatic structure is defined, one may declare:

\[
\mathbb{R} \text{ is a field}, \qquad
\mathbb{Z} \text{ is a ring}, \qquad
\mathbb{R}^n \text{ is a vector space over } \mathbb{R}.
\]

In Kleis:

\begin{verbatim}
implements Field(ℝ) {
    element zero = 0
    element one  = 1
    operation (+) = builtin_add
    operation (×) = builtin_mul
}
\end{verbatim}

This provides a \emph{model} for the field axioms using the actual real numbers.

Thus \texttt{implements} corresponds to constructing an algebra over a given
signature, or verifying that a concrete structure satisfies the axioms of an
abstract one.

%%%%%%%%%%%%%%%%%%%%%%%%%%%%%%%%%%%%%%%%%%%%%%%%%%%%%%%%%%%%%%%
\section{The \texttt{where} Construct: Logical Constraints}

Mathematicians frequently impose side-conditions:

\[
\text{Let $G$ be a group where $|G|$ is prime},
\]
\[
\text{Define $\det(A)$ where $A$ is a square matrix}.
\]

Kleis expresses such constraints through \texttt{where} clauses:

\begin{verbatim}
operation det : Matrix(n, n) → ℝ
\end{verbatim}

implicitly includes the logical condition “where matrix is square.”

More generally, Kleis may express:

\begin{verbatim}
implements SomeStructure(T)
    where Commutative(T)
{
    ...
}
\end{verbatim}

This states:

\[
\text{This interpretation of the structure is valid only when } T
\text{ satisfies the given property.}
\]

Thus \texttt{where} acts as a predicate restricting admissible models,
analogous to side-conditions in theorems.

%%%%%%%%%%%%%%%%%%%%%%%%%%%%%%%%%%%%%%%%%%%%%%%%%%%%%%%%%%%%%%%
\section{Nested Structures: Substructures Within Structures}

A ring contains two internal algebraic structures:

\[
(R, +, 0) \text{ is an Abelian group}, \qquad
(R, \cdot, 1) \text{ is a monoid}.
\]

Kleis expresses this literally:

\begin{verbatim}
structure Ring(R) {
    structure additive : AbelianGroup(R) { ... }
    structure multiplicative : Monoid(R)  { ... }
}
\end{verbatim}

Thus nested structures formalize the mathematical idea that an object may
possess multiple compatible algebraic structures.

This reflects the treatment of algebraic objects in Bourbaki and universal
algebra, where a single underlying set supports multiple signatures.

%%%%%%%%%%%%%%%%%%%%%%%%%%%%%%%%%%%%%%%%%%%%%%%%%%%%%%%%%%%%%%%
\section{Summary Table}

\begin{center}
\begin{tabular}{|c|c|c|}
\hline
\textbf{Kleis Construct} & \textbf{Mathematical Meaning} & \textbf{Example} \\
\hline
\texttt{extends} & Inheritance of axioms & Group extends Monoid \\
\hline
\texttt{over} & Parameterized structure & Vector space over a field \\
\hline
\texttt{implements} & Concrete model of axioms & $\mathbb{R}$ is a field \\
\hline
\texttt{where} & Logical side-condition & $\det(A)$ where $A$ is square \\
\hline
Nested structures & Internal algebraic components & Ring's additive/multiplicative parts \\
\hline
\end{tabular}
\end{center}

%%%%%%%%%%%%%%%%%%%%%%%%%%%%%%%%%%%%%%%%%%%%%%%%%%%%%%%%%%%%%%%
\section*{Conclusion}

Kleis mirrors mathematical practice in a direct, structural way.
Its constructs are not ad hoc language features but correspond
precisely to the methods by which mathematicians build and relate
algebraic objects.

Understanding Kleis as a language of abstract structures and their
concrete models provides a natural bridge between symbolic algebra,
formal logic, and executable mathematics.

\end{document}