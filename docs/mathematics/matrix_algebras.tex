\section{Matrix Algebras}

Matrices provide concrete realizations of linear operators and furnish
one of the most fundamental examples of associative algebras.  This
section introduces matrices over a field, their algebraic structure,
and their connection to linear maps.  Throughout, $F$ denotes a field.

\subsection{Matrices Over a Field}

\begin{definition}[Matrix]
    Let $m,n\in\mathbb{N}$.
    An \emph{$m\times n$ matrix over $F$} is a function
    \[
        A : \{1,\dots,m\} \times \{1,\dots,n\} \to F,
    \]
    written $A = (a_{ij})$ where $a_{ij} = A(i,j)$.
    The set of all such matrices is denoted $M_{m,n}(F)$.
\end{definition}

\begin{definition}[Matrix addition and scalar multiplication]
    For $A,B\in M_{m,n}(F)$ and $\lambda\in F$, define
    \[
        (A+B)_{ij} = a_{ij} + b_{ij},
        \qquad
        (\lambda A)_{ij} = \lambda a_{ij}.
    \]
    Then $M_{m,n}(F)$ is a vector space over $F$.
\end{definition}

\subsection{Matrix Multiplication}

\begin{definition}[Matrix product]
    Let $A\in M_{m,n}(F)$ and $B\in M_{n,p}(F)$.
    The \emph{matrix product} $AB\in M_{m,p}(F)$ is defined by
    \[
        (AB)_{ik} = \sum_{j=1}^n a_{ij} b_{jk}.
    \]
\end{definition}

\begin{proposition}
    Matrix multiplication is associative:
    \[
        (AB)C = A(BC).
    \]
\end{proposition}

\begin{definition}[Identity matrix]
    For each $n$, the \emph{identity matrix} $I_n\in M_n(F)$ is given by
    \[
        (I_n)_{ij} = \begin{cases}
                         1, & i=j,\\
                         0, & i\neq j.
        \end{cases}
    \]
\end{definition}

\begin{remark}
    Under matrix multiplication, $M_n(F)$ is a unital associative algebra.
\end{remark}

\subsection{Linear Maps and Matrices}

\begin{definition}[Linear operator]
    Let $V$ be an $n$-dimensional vector space over $F$ with basis
    $(e_1,\dots,e_n)$.
    A linear map $T:V\to V$ is represented by a unique matrix
    $[T]\in M_n(F)$ satisfying
    \[
        T(e_j) = \sum_{i=1}^n a_{ij} e_i.
    \]
\end{definition}

\begin{proposition}
    Composition of linear maps corresponds to matrix multiplication:
    \[
        [T\circ S] = [T]\,[S].
    \]
\end{proposition}

\subsection{Matrix Units and Structure Constants}

\begin{definition}[Matrix units]
    For $1\le i,j\le n$, the \emph{matrix unit} $E_{ij}$ is the matrix with
    \[
        (E_{ij})_{kl} = \delta_{ik}\delta_{jl}.
    \]
    These form a basis of $M_n(F)$ as a vector space.
\end{definition}

\begin{proposition}
    The multiplication of matrix units is given by
    \[
        E_{ij} E_{kl} = \delta_{jk} E_{il}.
    \]
\end{proposition}

Thus the algebra $M_n(F)$ is completely described by these structure
constants.

\subsection{Determinant and Trace}

\begin{definition}[Determinant]
    For an $n\times n$ matrix $A=(a_{ij})$, the \emph{determinant} is
    \[
        \det(A) =
        \sum_{\sigma\in S_n}
        \mathrm{sgn}(\sigma)
        \prod_{i=1}^n a_{i,\sigma(i)}.
    \]
\end{definition}

\begin{definition}[Trace]
    The \emph{trace} of $A\in M_n(F)$ is
    \[
        \mathrm{tr}(A) = \sum_{i=1}^n a_{ii}.
    \]
\end{definition}

\begin{proposition}
    For all $A,B\in M_n(F)$:
    \[
        \mathrm{tr}(AB) = \mathrm{tr}(BA).
    \]
\end{proposition}

\subsection{Invertible Matrices and Linear Groups}

\begin{definition}[General linear group]
    A matrix $A\in M_n(F)$ is \emph{invertible} if there exists $B$ such
    that $AB = BA = I_n$.  The set of all invertible matrices is the group
    \[
        GL(n,F).
    \]
\end{definition}

\begin{definition}[Special linear group]
    The subgroup
    \[
        SL(n,F) = \{\, A\in GL(n,F) : \det(A)=1 \,\}
    \]
    is the \emph{special linear group}.
\end{definition}

\subsection{Block Matrices}

\begin{definition}[Block matrix]
    Let $A_{ij}$ be matrices of appropriate sizes.
    A \emph{block matrix} is written
    \[
        A = \begin{pmatrix}
                A_{11} & A_{12} \\
                A_{21} & A_{22}
        \end{pmatrix}.
    \]
    Block addition and multiplication follow the usual rules of matrix
    algebra.
\end{definition}

\begin{remark}
    Block matrices allow the direct representation of tensor-product
    structures, direct sums of representations, and nested linear systems.
\end{remark}

\subsection{Matrix Algebras as Lie Algebras}

\begin{definition}[Commutator]
    The commutator bracket on $M_n(F)$ is
    \[
        [A,B] = AB - BA.
    \]
\end{definition}

\begin{proposition}
    $(M_n(F),[\cdot,\cdot])$ is a Lie algebra.
\end{proposition}

\begin{example}
    The subspace of trace-zero matrices
    \[
        \mathfrak{sl}_n(F)
        = \{\, A\in M_n(F) : \mathrm{tr}(A)=0 \,\}
    \]
    is a Lie subalgebra.
\end{example}

\subsection{Interpretation in Kleis}

Matrix structures are expressed in Kleis by parametric algebraic
interfaces such as:

\begin{verbatim}
structure Matrix(m: Nat, n: Nat, T) {
  operation transpose : Matrix(m,n,T) → Matrix(n,m,T)
}
\end{verbatim}

\begin{verbatim}
structure MatrixMultipliable(m: Nat, n: Nat, p: Nat, T) {
  operation multiply :
    Matrix(m,n,T) → Matrix(n,p,T) → Matrix(m,p,T)
}
\end{verbatim}

\begin{verbatim}
structure SquareMatrix(n: Nat, T) {
  operation det   : Matrix(n,n,T) → T
  operation trace : Matrix(n,n,T) → T
}
\end{verbatim}

These correspond directly to the linear, associative, and Lie-algebraic
properties of matrix spaces over a field.
