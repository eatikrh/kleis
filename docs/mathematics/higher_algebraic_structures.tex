\documentclass[12pt]{article}
\usepackage[utf8]{inputenc}
\usepackage{amsmath,amssymb,amsthm}
\usepackage{geometry}
\geometry{margin=1in}

\newtheorem{definition}{Definition}
\newtheorem{proposition}{Proposition}
\newtheorem{example}{Example}
\newtheorem{remark}{Remark}

\title{\textbf{Higher Algebraic Structures}\\[4pt]
\large Groups, Rings, Fields, and Vector Spaces}
\author{Kleis Language Project}
\date{}

\begin{document}
\maketitle

\section{Higher Algebraic Structures: Groups, Rings, Fields, and Vector Spaces}

This section extends the algebraic hierarchy beyond monoids.  The
presentation follows the classical structural method of Bourbaki:
each structure is defined by a carrier set together with operations
and equational axioms.

\subsection{Groups}

\begin{definition}[Group]
A \emph{group} is a triple $(G,\cdot,e)$ where:
\begin{enumerate}
  \item $(G,\cdot,e)$ is a monoid;
  \item every element $x\in G$ admits a (two-sided) inverse, i.e.,
        there exists $x^{-1}\in G$ such that
        \[
           x^{-1}\cdot x = e
           \qquad\text{and}\qquad
           x\cdot x^{-1} = e .
        \]
\end{enumerate}
\end{definition}

\begin{example}
The integers $(\mathbb{Z},+,0)$ form a group, where $x^{-1}=-x$.
\end{example}

\begin{definition}[Abelian group]
A group is \emph{abelian} if its multiplication is commutative:
\[
  x \cdot y = y \cdot x
  \qquad\text{for all } x,y\in G .
\]
\end{definition}

\begin{example}
$(\mathbb{Z},+)$ is abelian; $(S_n,\circ)$, the symmetric group,
is not.
\end{example}

\subsection{Rings}

\begin{definition}[Ring]
A \emph{ring} is a triple $(R,+,\times)$ such that:
\begin{enumerate}
  \item $(R,+)$ is an abelian group, with identity $0$ and inverse
        $-x$;
  \item $(R,\times)$ is a monoid, with identity $1$;
  \item multiplication distributes over addition:
        \[
           x\times (y+z) = (x\times y) + (x\times z),
           \qquad
           (x+y)\times z = (x\times z) + (y\times z).
        \]
\end{enumerate}
\end{definition}

\begin{remark}
A ring need not be commutative under multiplication; commutative rings
form a distinguished subclass.
\end{remark}

\begin{example}
The integers $(\mathbb{Z},+,\times)$ form a commutative ring.
The $n\times n$ matrices with real entries form a (noncommutative)
ring under matrix addition and multiplication.
\end{example}

\subsection{Fields}

\begin{definition}[Field]
A \emph{field} is a commutative ring $(F,+,\times)$ in which every
nonzero element admits a multiplicative inverse.  Thus:
\[
  x \neq 0 \implies \exists\,x^{-1}\in F \text{ such that }
  x\times x^{-1} = 1 .
\]
\end{definition}

\begin{example}
$\mathbb{Q}$, $\mathbb{R}$, and $\mathbb{C}$ are fields.  The integers
are not a field, since only $\pm 1$ are invertible.
\end{example}

\begin{remark}
Fields form the algebraic basis of linear algebra and support division
and scalar multiplication in vector spaces.
\end{remark}

\subsection{Vector Spaces}

\begin{definition}[Vector space]
Let $F$ be a field.  A \emph{vector space} over $F$ is a pair $(V,+)$
together with a scalar-multiplication operation
\[
  \cdot : F \times V \to V
\]
such that:
\begin{enumerate}
  \item $(V,+)$ is an abelian group with identity $0_v$;
  \item scalar multiplication satisfies:
        \begin{align*}
           a\cdot (v+w) &= (a\cdot v) + (a\cdot w), \\
           (a+b)\cdot v &= (a\cdot v) + (b\cdot v), \\
           (ab)\cdot v &= a\cdot(b\cdot v), \\
           1\cdot v &= v,
        \end{align*}
        for all $a,b\in F$ and $v,w\in V$.
\end{enumerate}
\end{definition}

\begin{example}
$\mathbb{R}^n$ is a vector space over the field $\mathbb{R}$.
Matrices of size $m\times n$ form a vector space over $\mathbb{R}$
under entrywise addition and scalar multiplication.
\end{example}

\subsection{Structural Relationships}

The hierarchy of algebraic structures can be summarized as:
\[
\text{Group} \supset \text{Monoid} \supset \text{Semigroup} \supset \text{Magma},
\]
and
\[
\text{Field} \supset \text{Commutative Ring} \supset \text{Ring}.
\]

Vector spaces are defined over fields:
\[
\text{Vector Space}(V) \text{ is defined over a Field } F.
\]

In categorical terms:
\begin{itemize}
  \item a monoid is a category with one object;
  \item a group is a groupoid with one object;
  \item a ring is a rig with additive inverses;
  \item a field is a commutative ring in which all nonzero arrows are invertible.
\end{itemize}

\subsection{Interpretation in Kleis}

The Kleis algebraic hierarchy mirrors these classical structures.  In
Kleis, each structure is specified by its operations and axioms:

\begin{verbatim}
structure Group(G) extends Monoid(G) {
  operation inv : G → G
  axiom left_inverse:
    ∀(x : G). inv(x) • x = e
}
\end{verbatim}

\begin{verbatim}
structure Ring(R) {
  structure additive : AbelianGroup(R)
  structure multiplicative : Monoid(R)
  axiom distributivity:
    ∀(x y z : R). x × (y + z) = (x × y) + (x × z)
}
\end{verbatim}

\begin{verbatim}
structure Field(F) extends Ring(F) {
  operation inverse : F → F
  axiom multiplicative_inverse:
    ∀(x : F) where x ≠ zero.
      inverse(x) × x = one
}
\end{verbatim}

\begin{verbatim}
structure VectorSpace(V) over Field(F) {
  operation (+) : V × V → V
  operation (·) : F × V → V
}
\end{verbatim}

\end{document}
