\documentclass[12pt]{article}
\usepackage{amsmath,amssymb}
\usepackage{mathpartir}
\usepackage{geometry}
\geometry{margin=1in}

\begin{document}

\section*{Algorithmic Redundancy Checking for Pattern Matching}

We present a \emph{syntax-directed}, algorithmic procedure for
detecting redundant (unreachable) branches in a single-scrutinee
\texttt{match} expression.

The key idea is to maintain a symbolic description of the set of
\emph{uncovered} values, and to test each new pattern against this
uncovered space.

\subsection*{Symbolic Value Shapes}

We introduce a notion of \emph{symbolic value shape}, or simply
\emph{shape}, which is a pattern-like term with no variables except
wildcards. These serve as canonical representatives of sets of values.

\[
\begin{array}{rcl}
q &::=& \_ \mid c \mid C(q_1,\dots,q_k)
\end{array}
\]

We will use finite sets of shapes
\(
  S = \{q_1,\dots,q_r\}
\)
to represent an over-approximation of the set of values of a given type
that are not yet covered by a set of patterns.

\subsection*{Uncovered-Space Judgment}

We define a judgment
\[
  \Delta \vdash \mathsf{Uncov}(T,\vec{p}) \Rightarrow S
\]
read:
\emph{under type context $\Delta$, the sequence of patterns
$\vec{p} = (p_1,\dots,p_n)$ for type $T$ leaves uncovered the symbolic
space $S$}.

Intuitively:
\begin{itemize}
  \item $S = \emptyset$ means the patterns are exhaustive.
  \item $S \neq \emptyset$ gives symbolic examples of values not matched.
\end{itemize}

We define $\mathsf{Uncov}$ inductively using an auxiliary judgment
\[
  \Delta \vdash \mathsf{Diff}(T, S, p) \Rightarrow S'
\]
which removes from $S$ all shapes covered by the pattern $p$.

\paragraph{Base Case}

With no patterns, the entire type $T$ is initially uncovered, represented
by a single wildcard shape:

\[
\inference[\textsc{U-Empty}]{}{
  \Delta \vdash \mathsf{Uncov}(T, ()) \Rightarrow \{\_\}
}
\]

\paragraph{Inductive Step}

To extend the uncovered space with a new pattern $p$:

\[
\inference[\textsc{U-Cons}]{
  \Delta \vdash \mathsf{Uncov}(T,(p_1,\dots,p_{k-1})) \Rightarrow S \\
  \Delta \vdash \mathsf{Diff}(T, S, p_k) \Rightarrow S'
}{
  \Delta \vdash \mathsf{Uncov}(T,(p_1,\dots,p_k)) \Rightarrow S'
}
\]

\subsection*{Difference Judgment \texorpdfstring{$\mathsf{Diff}$}{Diff}}

The judgment
\[
  \Delta \vdash \mathsf{Diff}(T, S, p) \Rightarrow S'
\]
means:
\emph{from the symbolic uncovered space $S$ of type $T$, remove all
shapes that are covered by pattern $p$, yielding $S'$}.

We define $\mathsf{Diff}$ syntax-directly on the structure of $p$ and
shapes $q \in S$.

\paragraph{Wildcard and Variable Patterns}

A wildcard or variable pattern matches \emph{all} values, so it removes
everything from the uncovered space:

\[
\inference[\textsc{D-Wild}]{}{
  \Delta \vdash \mathsf{Diff}(T, S, \_) \Rightarrow \emptyset
}
\]

\[
\inference[\textsc{D-Var}]{}{
  \Delta \vdash \mathsf{Diff}(T, S, x) \Rightarrow \emptyset
}
\]

\paragraph{Constant Patterns}

A constant pattern $c$ removes only shapes compatible with $c$.
For simplicity we say: any wildcard shape \(\_\) is split into
\emph{one} shape equal to $c$ and \emph{one} residual wildcard representing
``all values except $c$''. This is schematic; in an implementation one
would refine the residual space more precisely.

We write
\(
  S = \{q_1,\dots,q_r\}
\)
and define:
\[
\inference[\textsc{D-Const}]{}{
  \Delta \vdash \mathsf{Diff}(T, \{c\}, c) \Rightarrow \emptyset
}
\]

\[
\inference[\textsc{D-Const-Wild}]{}{
  \Delta \vdash \mathsf{Diff}(T, \{\_\}, c) \Rightarrow \{\_\}'
}
\]

Here $\{\_\}'$ is understood informally as a wildcard shape over the
residual values of $T$ distinct from $c$. For non-matching shapes,
we propagate them unchanged:

\[
\inference[\textsc{D-Const-Other}]{}{
  \Delta \vdash \mathsf{Diff}(T, \{q\}, c) \Rightarrow \{q\}
  \quad\text{if $q$ is incompatible with $c$}
}
\]

For a general finite set $S$, we apply $\mathsf{Diff}$ pointwise and
union the results:

\[
\inference[\textsc{D-Set}]{
  \forall i.\ \Delta \vdash \mathsf{Diff}(T, \{q_i\}, p) \Rightarrow S_i
}{
  \Delta \vdash \mathsf{Diff}(T, \{q_1,\dots,q_r\}, p)
    \Rightarrow \bigcup_{i=1}^r S_i
}
\]

\paragraph{Constructor Patterns}

Assume $T$ is an algebraic data type
\[
  \texttt{data } T = C_1(\vec{\tau}_1) \mid \cdots \mid C_n(\vec{\tau}_n).
\]

A constructor pattern $C(\vec{p})$ only affects shapes whose outer
constructor is $C$ or wildcard \(\_\). Wildcards are refined into
constructor-specific shapes.

For a single shape:
\[
\inference[\textsc{D-Constr-Match}]{
  q = C(q_1,\dots,q_k) \\
  \Delta \vdash \mathsf{DiffArgs}((\tau_1,\dots,\tau_k),
     (q_1,\dots,q_k), (\vec{p}))
    \Rightarrow
     \{(q'_1,\dots,q'_k)_1,\dots,(q'_1,\dots,q'_k)_m\}
}{
  \Delta \vdash \mathsf{Diff}(T, \{q\}, C(\vec{p}))
    \Rightarrow
      \{\,C((q'_1,\dots,q'_k)_1),\dots,C((q'_1,\dots,q'_k)_m)\,\}
}
\]

Here $\mathsf{DiffArgs}$ is an auxiliary, syntax-directed procedure
that subtracts the argument-pattern space $(\vec{p})$ from the
argument-shape tuple $(q_1,\dots,q_k)$ componentwise; details depend
on the desired precision of the checker.

A wildcard shape \(\_\) is expanded into constructor-specific shapes
before applying \textsc{D-Constr-Match}:

\[
\inference[\textsc{D-Constr-Wild}]{
  \text{Constructors of $T$ are exactly } C_1,\dots,C_n \\
  \forall i.\ q_i = C_i(\underbrace{\_,\dots,\_}_{k_i})
  \\
  \Delta \vdash \mathsf{Diff}(T, \{q_i\}, C(\vec{p})) \Rightarrow S_i
}{
  \Delta \vdash \mathsf{Diff}(T, \{\_\}, C(\vec{p}))
    \Rightarrow \bigcup_i S_i
}
\]

Shapes whose outer constructor is different from $C$ are unaffected:

\[
\inference[\textsc{D-Constr-Other}]{}{
  q = C'(q_1,\dots,q_\ell),\ C' \neq C
}{
  \Delta \vdash \mathsf{Diff}(T, \{q\}, C(\vec{p}))
    \Rightarrow \{q\}
}
\]

\subsection*{Algorithmic Usefulness and Redundancy}

Given the uncovered-space analysis, we define a purely syntactic
\emph{usefulness} judgment for patterns:

\[
  \Delta \vdash \mathsf{Useful}(T,\vec{p},p)
\]

read: \emph{for type $T$, pattern $p$ is useful (non-redundant) relative
to the preceding patterns $\vec{p}$}.

Algorithmically:

\[
\inference[\textsc{U-Alg}]{
  \Delta \vdash \mathsf{Uncov}(T,\vec{p}) \Rightarrow S \\
  \exists q \in S,\ \exists \theta.\ \textsf{match}(p,q) = \theta
}{
  \Delta \vdash \mathsf{Useful}(T,\vec{p},p)
}
\]

That is, $p$ is useful if there is some symbolic shape $q$ in the current
uncovered space $S$ that it still matches.

Dually, we define an algorithmic redundancy judgment:

\[
  \Delta \vdash \mathsf{Redundant}(T,\vec{p},p)
\]

when \emph{no uncovered shape} is matched by $p$:

\[
\inference[\textsc{R-Alg}]{
  \Delta \vdash \mathsf{Uncov}(T,\vec{p}) \Rightarrow S \\
  \forall q \in S.\ \textsf{match}(p,q) \text{ is undefined}
}{
  \Delta \vdash \mathsf{Redundant}(T,\vec{p},p)
}
\]

\subsection*{Algorithmic Non--Redundancy for a Full Branch List}

Given a full branch list $\vec{p} = (p_1,\dots,p_m)$, we say it is
algorithmically non--redundant for $T$ if:

\[
  \Delta \vdash \mathsf{NR}^{\mathsf{alg}}(T,(p_1,\dots,p_m))
\]

holds, where:

\[
\inference[\textsc{NR-Alg-Empty}]{}{
  \Delta \vdash \mathsf{NR}^{\mathsf{alg}}(T,())
}
\]

\[
\inference[\textsc{NR-Alg-Cons}]{
  \Delta \vdash \mathsf{NR}^{\mathsf{alg}}(T,(p_1,\dots,p_{k-1})) \\
  \Delta \vdash \mathsf{Useful}(T,(p_1,\dots,p_{k-1}),p_k)
}{
  \Delta \vdash \mathsf{NR}^{\mathsf{alg}}(T,(p_1,\dots,p_k))
}
\]

Thus each new pattern $p_k$ must be \emph{useful} with respect to the
uncovered space after the previous patterns.

\subsection*{Meta--Properties (Informal)}

\begin{itemize}
  \item \textbf{Soundness}: If
        $\Delta \vdash \mathsf{Redundant}(T,\vec{p},p)$ then
        $p$ is semantically redundant in the sense of the semantic
        non--redundancy judgment $\mathsf{NR}$ (no value of type $T$ is
        matched first by $p$).

  \item \textbf{(Possible) Completeness}: With a sufficiently precise
        definition of $\mathsf{Diff}$ and $\mathsf{DiffArgs}$, the
        algorithmic notion $\mathsf{Useful}$ can be made complete with
        respect to the semantic notion of non--redundancy for
        first-order algebraic data types, as in classical coverage
        algorithms for ML/Haskell pattern matching.
\end{itemize}

In practice, Kleis can implement $\mathsf{Uncov}$ and $\mathsf{Useful}$
as a coverage-checking pass on pattern matrices, following techniques
à la Maranget, using the rules above as a specification.

\end{document}