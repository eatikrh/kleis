\section{Lie Groups and the Exponential Map}

Lie groups provide a unification of algebraic and geometric ideas.
They are groups whose elements form a smooth manifold, and whose group
operations are differentiable maps.  Their tangent spaces at the
identity carry the structure of a Lie algebra, allowing differential
methods to study global group behavior.

\subsection{Lie Groups}

\begin{definition}[Lie group]
    A \emph{Lie group} is a smooth manifold $G$ equipped with a group
    structure such that both the multiplication
    \[
        m : G \times G \to G, \qquad m(x,y)=xy,
    \]
    and the inversion map
    \[
        \iota : G \to G, \qquad \iota(x)=x^{-1},
    \]
    are smooth functions.
\end{definition}

\begin{example}
    The groups $GL(n,\mathbb{R})$ and $SL(n,\mathbb{R})$ of invertible and
    special linear matrices are Lie groups.  Their manifold structure is
    inherited from the open subset of $\mathbb{R}^{n^2}$ consisting of
    matrices with nonzero determinant.
\end{example}

\subsection{Lie Algebra of a Lie Group}

\begin{definition}[Lie algebra of $G$]
    Let $G$ be a Lie group.
    The \emph{Lie algebra} $\mathfrak{g}$ of $G$ is the tangent space at the
    identity element:
    \[
        \mathfrak{g} = T_e G.
    \]
\end{definition}

\begin{definition}[Bracket]
    For $X,Y \in \mathfrak{g}$, the \emph{Lie bracket}
    is defined by
    \[
        [X,Y] = (XY - YX)\big|_e,
    \]
    i.e.\ the commutator of left-invariant vector fields evaluated at the
    identity.
\end{definition}

\begin{proposition}
    $(\mathfrak{g},[\cdot,\cdot])$ is a Lie algebra.
\end{proposition}

\begin{example}
    For $G = GL(n,\mathbb{R})$, the Lie algebra is
    \[
        \mathfrak{g} = \mathfrak{gl}_n(\mathbb{R})
        = M_n(\mathbb{R}),
    \]
    with the commutator bracket $[A,B] = AB - BA$.
\end{example}

\subsection{One-Parameter Subgroups}

\begin{definition}[One-parameter subgroup]
    A smooth homomorphism
    \[
        \gamma : \mathbb{R} \to G
    \]
    satisfying $\gamma(s+t) = \gamma(s)\gamma(t)$
    is called a \emph{one-parameter subgroup}.
\end{definition}

\begin{proposition}
    Each $X\in\mathfrak{g}$ determines a unique one-parameter subgroup via
    the differential equation
    \[
        \gamma'(0) = X.
    \]
    The solution is given by the exponential map below.
\end{proposition}

\subsection{The Exponential Map}

\begin{definition}[Matrix exponential]
    For $A \in M_n(\mathbb{R})$, the \emph{matrix exponential} is
    \[
        \exp(A) = \sum_{k=0}^{\infty} \frac{A^k}{k!}.
    \]
\end{definition}

\begin{definition}[Exponential map of a Lie group]
    For a general Lie group $G$, the \emph{exponential map} is the smooth
    map
    \[
        \exp : \mathfrak{g} \to G
    \]
    defined by
    \[
        \exp(X) = \gamma_X(1),
    \]
    where $\gamma_X$ is the one-parameter subgroup generated by $X$.
\end{definition}

\begin{proposition}[Properties of the exponential map]
    For any Lie group $G$:
    \begin{enumerate}
        \item $\exp(0) = e$.
        \item $\exp(tX)\exp(sX)=\exp((t+s)X)$.
        \item The differential $(d\exp)_0$ is the identity on $\mathfrak{g}$.
        \item Near $0\in\mathfrak{g}$, $\exp$ is a local diffeomorphism.
    \end{enumerate}
\end{proposition}

\begin{example}
    For $G = GL(n,\mathbb{R})$, the exponential is the usual matrix
    exponential, and for sufficiently small $A$,
    \[
        \exp : \mathfrak{gl}_n(\mathbb{R}) \to GL(n,\mathbb{R})
    \]
    is a diffeomorphism onto a neighborhood of $I$.
\end{example}

\subsection{Adjoint Representation}

\begin{definition}[Adjoint action]
    The \emph{adjoint action} of $G$ on itself is
    \[
        \mathrm{Ad}_g(X)
        = gXg^{-1}
        \quad \text{(in matrix groups)}.
    \]
    In general, $\mathrm{Ad}_g$ is the differential at $e$ of the
    conjugation map $c_g(x) = gxg^{-1}$.
\end{definition}

\begin{definition}[Adjoint representation of the Lie algebra]
    The induced map on the Lie algebra is
    \[
        \mathrm{ad}_X(Y) = [X,Y].
    \]
\end{definition}

\begin{proposition}
    $\mathrm{ad} : \mathfrak{g} \to \mathfrak{gl}(\mathfrak{g})$ is a Lie
    algebra homomorphism.
\end{proposition}

\subsection{Exponential of a Commutator}

\begin{proposition}[Baker--Campbell--Hausdorff formula (local)]
    For $X,Y$ sufficiently small in $\mathfrak{g}$,
    \[
        \exp(X)\exp(Y)
        = \exp\!\bigl(X + Y + \tfrac12[X,Y]
        + \text{higher commutators}\bigr).
    \]
\end{proposition}

This expresses the local group structure in purely algebraic terms.

\subsection{Interpretation in Kleis}

Kleis expresses these structures through parametric and nested algebraic
structures:

\begin{verbatim}
structure LieAlgebra(g) extends VectorSpace(g) {
  operation bracket : g × g → g
  axiom antisymmetry:   [x,y] = -[y,x]
  axiom jacobi:         [x,[y,z]] + [y,[z,x]] + [z,[x,y]] = zero
}
\end{verbatim}

\begin{verbatim}
structure LieGroup(G) {
  element identity : G
  operation multiply : G × G → G
  operation inverse  : G → G
  operation exp      : Tangent(G) → G
}
\end{verbatim}

Matrix Lie groups such as $GL(n,\mathbb{R})$ arise from specialized
instances where:

\begin{verbatim}
implements LieGroup(Matrix(n,n,ℝ)) {
  operation exp = builtin_matrix_exponential
}
\end{verbatim}

Thus Kleis captures both the algebraic and differential aspects of Lie
theory.
