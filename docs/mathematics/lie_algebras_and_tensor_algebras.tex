\section{Lie Algebras and Tensor Algebras}

This section develops two fundamental algebraic structures which lie
above vector spaces and associative algebras: Lie algebras, which
capture infinitesimal symmetry, and tensor algebras, which generate
the free associative algebra over a vector space.  Both constructions
play a central role in geometry, representation theory, and modern
mathematical physics.

\subsection{Lie Algebras}

\begin{definition}[Lie algebra]
    Let $F$ be a field.
    A \emph{Lie algebra} over $F$ is a vector space $\frak{g}$
    together with a bilinear map
    \[
        [\,\cdot\,,\,\cdot\,] : \frak{g} \times \frak{g} \to \frak{g},
    \]
    called the \emph{Lie bracket}, satisfying:
    \begin{enumerate}
        \item \textbf{Antisymmetry:}
        \[
            [x,y] = -[y,x]
            \qquad\text{for all } x,y\in \frak{g};
        \]
        \item \textbf{Jacobi identity:}
        \[
            [x,[y,z]] + [y,[z,x]] + [z,[x,y]] = 0
            \qquad\text{for all } x,y,z\in \frak{g}.
        \]
    \end{enumerate}
\end{definition}

\begin{example}
    Let $A$ be an associative algebra over $F$.
    The commutator
    \[
        [x,y] = xy - yx
    \]
    defines a Lie algebra structure on the underlying vector space of $A$.
\end{example}

\begin{example}
    The vector space of $n\times n$ matrices $\mathfrak{gl}_n(F)$ with
    the commutator bracket is a Lie algebra.
    The subspace of trace-zero matrices $\mathfrak{sl}_n(F)$ is also a Lie algebra.
\end{example}

\subsubsection*{Structure Constants}

Let $(e_1,\dots,e_n)$ be a basis of $\frak{g}$.
Then the bracket is determined by constants $c^{k}_{ij}\in F$ such that
\[
    [e_i,e_j] = \sum_{k=1}^n c^{k}_{ij} e_k.
\]
The Jacobi identity is equivalent to the relations
\[
    \sum_{m=1}^n
    \bigl( c^{m}_{ij} c^{\ell}_{mk}
    +c^{m}_{jk} c^{\ell}_{mi}
    +c^{m}_{ki} c^{\ell}_{mj} \bigr)
    = 0
    \qquad\text{for all } i,j,k,\ell.
\]

\subsection{Representations of Lie Algebras}

\begin{definition}[Representation]
    Let $\frak{g}$ be a Lie algebra and $V$ a vector space over $F$.
    A \emph{representation} of $\frak{g}$ on $V$ is a linear map
    \[
        \rho : \frak{g} \to \mathrm{End}(V)
    \]
    such that
    \[
        \rho([x,y]) = [\,\rho(x), \rho(y)\,]
        = \rho(x)\rho(y) - \rho(y)\rho(x).
    \]
\end{definition}

\begin{example}
    The \emph{adjoint representation} is defined by
    \[
        \mathrm{ad} : \frak{g} \to \mathrm{End}(\frak{g}),
        \qquad
        \mathrm{ad}(x)(y) = [x,y].
    \]
\end{example}

\subsection{Tensor Algebras}

\begin{definition}[Tensor algebra]
    Let $V$ be a vector space over $F$.
    The \emph{tensor algebra} of $V$ is the graded vector space
    \[
        T(V) = \bigoplus_{n\ge 0} V^{\otimes n},
    \]
    with $V^{\otimes 0} = F$,
    equipped with the associative multiplication
    \[
        (v_1\otimes \cdots \otimes v_m)
        \; * \;
        (w_1\otimes \cdots \otimes w_n)
        \;=\;
        v_1\otimes \cdots \otimes v_m \otimes
        w_1\otimes \cdots \otimes w_n.
    \]
\end{definition}

\begin{remark}
    $T(V)$ is the \emph{free associative algebra} generated by $V$.
    That is, for any associative algebra $A$ and linear map
    $\varphi:V\to A$, there exists a unique algebra homomorphism
    \[
        \widetilde{\varphi} : T(V) \to A
    \]
    extending $\varphi$.
\end{remark}

\subsection{Exterior and Symmetric Algebras}

The tensor algebra contains two fundamental quotient algebras:

\begin{definition}[Exterior algebra]
    The \emph{exterior algebra} $\Lambda(V)$ is the quotient of $T(V)$ by the ideal generated by
    \[
        v\otimes v \qquad (v\in V).
    \]
    The induced product is denoted $\wedge$ and is antisymmetric.
\end{definition}

\begin{definition}[Symmetric algebra]
    The \emph{symmetric algebra} $S(V)$ is the quotient of $T(V)$ by the ideal generated by
    \[
        v\otimes w - w\otimes v.
    \]
    It is the free commutative algebra generated by $V$.
\end{definition}

\begin{remark}
    These constructions underlie multilinear algebra, differential forms,
    and polynomial algebras.  They also serve as foundations for Clifford
    algebras and spin geometry.
\end{remark}

\subsection{The Universal Enveloping Algebra}

\begin{definition}[Universal enveloping algebra]
    For a Lie algebra $\frak{g}$, the \emph{universal enveloping algebra}
    $U(\frak{g})$ is the quotient of the tensor algebra $T(\frak{g})$
    by the ideal generated by the relations
    \[
        x\otimes y - y\otimes x - [x,y].
    \]
\end{definition}

\begin{remark}
    $U(\frak{g})$ provides the bridge between Lie algebras and associative
    algebras, and plays a central role in representation theory.
\end{remark}

\subsection{Interpretation in Kleis}

The structures above naturally admit the following Kleis specifications:

\begin{verbatim}
structure LieAlgebra(g) over Field(F) {
  operation bracket : g × g → g
  axiom antisymmetry:
    ∀(x y : g). bracket(x,y) = - bracket(y,x)
  axiom jacobi:
    ∀(x y z : g).
      bracket(x, bracket(y,z))
    + bracket(y, bracket(z,x))
    + bracket(z, bracket(x,y)) = 0
}
\end{verbatim}

\begin{verbatim}
structure TensorAlgebra(T) over VectorSpace(V) {
  operation tensor : V × V → V⊗V
  operation multiply : T × T → T
  axiom associativity:
    ∀(a b c : T). multiply(multiply(a,b),c)
                 = multiply(a,multiply(b,c))
}
\end{verbatim}

These structures provide the mathematical foundations for differential
operators, representation theory, and multilinear constructions within
the Kleis language.
