\documentclass[12pt]{article}
\usepackage[utf8]{inputenc}
\usepackage{amsmath,amssymb,amsthm}
\usepackage{geometry}
\geometry{margin=1in}

\newtheorem{definition}{Definition}
\newtheorem{proposition}{Proposition}
\newtheorem{example}{Example}
\newtheorem{remark}{Remark}

\title{\textbf{Even Higher Algebraic Structures}\\[4pt]
\large Modules, Algebras, Group Actions, and Representations}
\author{Kleis Language Project}
\date{}

\begin{document}
\maketitle

\section{Higher Algebraic Structures: Modules, Algebras, Group Actions, and Representations}

This section develops the next layer of abstract algebra above groups,
rings, fields, and vector spaces.  The structures introduced here
provide the algebraic foundation for linear algebra, representation
theory, and many constructions appearing in the semantic foundations
of Kleis.

\subsection{Modules Over Rings}

\begin{definition}[$R$-module]
Let $R$ be a ring.  An \emph{$R$-module} is an abelian group $(M,+)$
together with a scalar multiplication
\[
  \cdot : R \times M \to M
\]
such that, for all $r,s\in R$ and $x,y\in M$,
\begin{align*}
  r\cdot(x+y) &= r\cdot x + r\cdot y, \\
  (r+s)\cdot x &= r\cdot x + s\cdot x, \\
  (rs)\cdot x &= r\cdot(s\cdot x), \\
  1\cdot x &= x,
\end{align*}
where $1$ denotes the multiplicative identity of $R$.
\end{definition}

\begin{remark}
Modules generalize vector spaces by allowing the scalars to come from
an arbitrary ring rather than a field.  When $R$ is a field, an
$R$-module is precisely a vector space over $R$.
\end{remark}

\subsection{Algebras Over a Ring or Field}

\begin{definition}[Algebra over a ring]
Let $R$ be a ring.  An \emph{$R$-algebra} is an $R$-module $(A,+,\cdot)$
together with a bilinear multiplication
\[
  * : A \times A \to A
\]
such that $(A,*,1)$ is a ring and scalar multiplication in $A$ is
compatible with multiplication:
\[
  r\cdot(a * b) = (r\cdot a) * b = a * (r\cdot b).
\]
\end{definition}

\begin{example}
\leavevmode
\begin{enumerate}
  \item The polynomial ring $R[x]$ is an $R$-algebra.
  \item The matrix ring $M_n(R)$ is an $R$-algebra.
  \item Over a field $F$, any associative algebra (Lie, Clifford,
        tensor, exterior, symmetric) is an $F$-algebra.
\end{enumerate}
\end{example}

\subsection{Group Actions}

\begin{definition}[Group action]
Let $G$ be a group and $X$ a set.  A \emph{(left) action} of $G$ on $X$
is a function
\[
  \alpha : G \times X \to X
\qquad\text{denoted } g\cdot x,
\]
such that, for all $g,h\in G$ and $x\in X$,
\begin{align*}
  e\cdot x &= x, \\
  (gh)\cdot x &= g\cdot(h\cdot x),
\end{align*}
where $e$ is the identity of $G$.
\end{definition}

\begin{remark}
A group action is a homomorphism $G \to \mathrm{Bij}(X)$, the group of
all bijections of $X$.
\end{remark}

\subsection{Representation Theory}

\begin{definition}[Linear representation]
Let $G$ be a group and $F$ a field.  A \emph{representation} of $G$
on a vector space $V$ over $F$ is a group homomorphism
\[
  \rho : G \to \mathrm{GL}(V),
\]
where $\mathrm{GL}(V)$ denotes the group of invertible linear
transformations of $V$.
\end{definition}

\begin{example}
\leavevmode
\begin{enumerate}
  \item The trivial representation: $\rho(g)=\mathrm{id}$.
  \item The permutation representation of $S_n$ on $\mathbb{R}^n$.
  \item The adjoint representation of a Lie group on its Lie algebra.
\end{enumerate}
\end{example}

\begin{remark}
Representations allow groups to be studied through the linear algebra
of vector spaces.  Much of modern algebra, geometry, and quantum
physics is based on the interaction between groups and their
representations.
\end{remark}

\subsection{Structural Overview}

The higher algebraic hierarchy extends the basic tower:
\[
  \text{Magma} \subset
  \text{Semigroup} \subset
  \text{Monoid} \subset
  \text{Group} \subset
  \text{Ring} \subset
  \text{Field}.
\]

Above these, one obtains:
\[
  \text{Field} \implies \text{Vector Space},
\qquad
  \text{Ring} \implies \text{Module},
\qquad
  \text{Module} \implies \text{$R$-Algebra}.
\]

Group actions and group representations provide the fundamental means
by which algebraic symmetry is expressed within linear structures.

\subsection{Interpretation in Kleis}

The structures introduced above admit direct specification in Kleis.
For instance:
\begin{verbatim}
structure Module(M) over Ring(R) {
  operation (+) : M × M → M
  operation (·) : R × M → M
}
\end{verbatim}

\begin{verbatim}
structure Algebra(A) over Ring(R) {
  structure module : Module(A) over Ring(R)
  operation (*) : A × A → A
  axiom bilinear:
    ∀(r : R, x y : A).
      r · (x * y) = (r · x) * y = x * (r · y)
}
\end{verbatim}

\begin{verbatim}
structure Representation(G, V) over Field(F) {
  operation act : G × V → V
  axiom compatibility:
    ∀(g h : G, v : V).
      act(g · h, v) = act(g, act(h, v))
}
\end{verbatim}

\end{document}
