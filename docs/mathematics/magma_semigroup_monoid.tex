\documentclass[12pt]{article}
\usepackage[utf8]{inputenc}
\usepackage{amsmath,amssymb,amsthm}
\usepackage{geometry}
\geometry{margin=1in}

\newtheorem{definition}{Definition}
\newtheorem{proposition}{Proposition}
\newtheorem{example}{Example}
\newtheorem{remark}{Remark}

\title{\textbf{Algebraic Foundations}\\[4pt]
\large Magma, Semigroup, and Monoid}
\author{Kleis Language Project}
\date{}

\begin{document}
\maketitle

\section{Algebraic Foundations: Magma, Semigroup, and Monoid}

This appendix recalls the universal–algebraic hierarchy underlying many
of the core abstractions in Kleis.  The presentation follows the
style of Bourbaki and classical algebraic specification.

\subsection{Magma}

\begin{definition}[Magma]
A \emph{magma} is a pair $(M,\star)$ consisting of a set $M$ together
with a binary operation
\[
  \star : M \times M \to M.
\]
No algebraic axioms are required: in particular, associativity and
identity need not hold.
\end{definition}

\begin{example}
Let $M=\{1,2,3\}$ and define $x \star y = x$ for all $x,y\in M$.
Then $(M,\star)$ is a magma which is neither associative nor admits
an identity element.
\end{example}

\subsection{Semigroup}

\begin{definition}[Semigroup]
A \emph{semigroup} is a magma $(S,\star)$ whose operation is
associative:
\[
  (x \star y)\star z = x \star (y \star z)
  \qquad\text{for all } x,y,z\in S.
\]
\end{definition}

\begin{remark}
A semigroup may fail to have a neutral element.  Thus the usual notion
of a \emph{fold} over an empty list is not available for arbitrary
semigroups.
\end{remark}

\subsection{Monoid}

\begin{definition}[Monoid]
A \emph{monoid} is a semigroup $(M,\cdot)$ together with a distinguished
element $e\in M$, called the \emph{identity}, such that
\[
  e \cdot x = x
  \qquad\text{and}\qquad
  x \cdot e = x
  \qquad\text{for all } x \in M.
\]
Thus a monoid is a triple $(M,\cdot,e)$ satisfying associativity and
both left and right identity laws.
\end{definition}

\begin{example}[Classical monoids]
\leavevmode
\begin{enumerate}
  \item $(\mathbb{N},+,0)$ is a monoid.
  \item $(\text{Strings},\text{concatenation},\varepsilon)$ is a monoid,
        where $\varepsilon$ is the empty string.
  \item For any set $X$, the endofunctions $X^X$ with composition form a
        monoid; the identity element is $\mathrm{id}_X$.
\end{enumerate}
\end{example}

\subsection{Relationship Between the Structures}

Every monoid is a semigroup, and every semigroup is a magma.  The
converse implications do not hold.  In summary:
\[
  \text{Monoid}
  \;\subset\;
  \text{Semigroup}
  \;\subset\;
  \text{Magma}.
\]

\subsection{Categorical Interpretation}

\begin{proposition}
A monoid is precisely a category with a single object.
\end{proposition}

\begin{proof}
Let $\mathcal{C}$ be a category with one object $*$.
Its morphisms $\mathrm{Hom}( *, * )$ carry a natural associative
composition, and the identity morphism $\mathrm{id}_*$ is a unit.
Conversely, given a monoid $(M,\cdot,e)$, construct a category with
one object whose endomorphisms are the elements of $M$ and whose
composition is $\cdot$.
\end{proof}

\subsection{Interpretation in Kleis}

The algebraic hierarchy of Kleis mirrors the universal–algebraic
definitions above.  In Kleis, the corresponding specifications are:

\begin{verbatim}
structure Magma(M) {
  operation (•) : M × M → M
}

structure Semigroup(S) extends Magma(S) {
  axiom associativity:
    ∀(x y z : S). (x • y) • z = x • (y • z)
}

structure Monoid(M) extends Semigroup(M) {
  element e : M
  axiom left_identity:  ∀(x : M). e • x = x
  axiom right_identity: ∀(x : M). x • e = x
}
\end{verbatim}

Thus the standard fold and accumulation operators become available
precisely at the level of monoids, in agreement with both algebraic and
category-theoretic semantics.
\end{document}
