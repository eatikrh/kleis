\documentclass[12pt]{article}
\usepackage{amsmath,amssymb}
\usepackage{mathpartir}
\usepackage{geometry}
\geometry{margin=1in}

\begin{document}

\section*{Operational Semantics of Kleis (Core Fragment)}

In this appendix we present a big--step operational semantics for a core
fragment of Kleis expressions.
The judgment
\[
  \rho \vdash e \Downarrow v
\]
is read: \emph{in environment $\rho$, expression $e$ evaluates to value $v$}.

\subsection*{Syntax (Core Fragment)}

We consider the following informal grammar for expressions:
\[
\begin{array}{rcl}
e &::=& x \mid v \mid \lambda x . e \mid e_1\,e_2 \mid
        \texttt{let } x = e_1 \texttt{ in } e_2 \\
  && \mid \texttt{if } e_0 \texttt{ then } e_1 \texttt{ else } e_2 \\
  && \mid \texttt{match } e \{ p_1 \Rightarrow e_1 \;\mid\; \cdots \;\mid\; p_n \Rightarrow e_n \}
\end{array}
\]

Values:
\[
\begin{array}{rcl}
v &::=& c \mid \lambda x . e \mid C(v_1,\dots,v_k)
\end{array}
\]
where $c$ is a primitive literal (number, boolean, string, etc.) and
$C$ ranges over data constructors of algebraic data types.

Patterns:
\[
\begin{array}{rcl}
p &::=& \_ \mid x \mid C(p_1,\dots,p_k) \mid c
\end{array}
\]

We assume a fixed set of primitive operations (\texttt{+}, \texttt{-}, etc.)
given by a meta--level function
\(\llbracket \text{op} \rrbracket(v_1,\dots,v_n)\).

\subsection*{Environments}

An \emph{environment} $\rho$ is a finite mapping from variables to values.
We write $\rho[x \mapsto v]$ for the environment that maps $x$ to $v$ and
agrees with $\rho$ on all other variables.

The lookup of a variable is written $\rho(x)$.

\subsection*{Values and Literals}

Literals evaluate to themselves:

\[
\inference[\textsc{E-Const}]{}{
  \rho \vdash c \Downarrow c
}
\]

Lambda abstractions are values:

\[
\inference[\textsc{E-Lam}]{}{
  \rho \vdash \lambda x . e \Downarrow \lambda x . e
}
\]

Data constructors applied to values evaluate directly to a constructor value:

\[
\inference[\textsc{E-Constr}]{}{
  \rho \vdash C(v_1,\dots,v_k) \Downarrow C(v_1,\dots,v_k)
}
\]

\subsection*{Variables}

Variables are looked up in the environment:

\[
\inference[\textsc{E-Var}]{}{
  \rho(x) = v
}{
  \rho \vdash x \Downarrow v
}
\]

\subsection*{Function Application}

We use call--by--value semantics.
To evaluate $e_1\,e_2$ we first evaluate $e_1$ to a function value,
then $e_2$ to an argument value, then evaluate the body in an extended environment.

\[
\inference[\textsc{E-App}]{
  \rho \vdash e_1 \Downarrow \lambda x . e \\
  \rho \vdash e_2 \Downarrow v_2 \\
  \rho[x \mapsto v_2] \vdash e \Downarrow v
}{
  \rho \vdash e_1\,e_2 \Downarrow v
}
\]

If $e_1$ evaluates to a non-lambda value, the semantics is undefined
(statically ruled out by typing).

\subsection*{\texttt{let}-Bindings}

A \texttt{let}--binding evaluates its right-hand side and then evaluates
the body under an extended environment:

\[
\inference[\textsc{E-Let}]{
  \rho \vdash e_1 \Downarrow v_1 \\
  \rho[x \mapsto v_1] \vdash e_2 \Downarrow v
}{
  \rho \vdash \texttt{let } x = e_1 \texttt{ in } e_2 \Downarrow v
}
\]

\subsection*{Conditionals}

Booleans are assumed to be primitive values \texttt{True}, \texttt{False}.

\[
\inference[\textsc{E-IfTrue}]{
  \rho \vdash e_0 \Downarrow \texttt{True} \\
  \rho \vdash e_1 \Downarrow v
}{
  \rho \vdash \texttt{if } e_0 \texttt{ then } e_1 \texttt{ else } e_2 \Downarrow v
}
\]

\[
\inference[\textsc{E-IfFalse}]{
  \rho \vdash e_0 \Downarrow \texttt{False} \\
  \rho \vdash e_2 \Downarrow v
}{
  \rho \vdash \texttt{if } e_0 \texttt{ then } e_1 \texttt{ else } e_2 \Downarrow v
}
\]

If $e_0$ evaluates to a non-boolean, the semantics is undefined (again,
ruled out statically).

\subsection*{Primitive Operations}

For a primitive $n$-ary operator $\text{op}(e_1,\dots,e_n)$:

\[
\inference[\textsc{E-Prim}]{
  \rho \vdash e_1 \Downarrow v_1 \quad \cdots \quad \rho \vdash e_n \Downarrow v_n \\
  \llbracket \text{op} \rrbracket(v_1,\dots,v_n) = v
}{
  \rho \vdash \text{op}(e_1,\dots,e_n) \Downarrow v
}
\]

Here $\llbracket \text{op} \rrbracket$ is a meta--level interpretation
of Kleis’s built-in arithmetic, logical, or matrix operations.

\subsection*{Pattern Matching}

We give the semantics of:

\[
\texttt{match } e \{ p_1 \Rightarrow e_1 \;|\; \dots \;|\; p_n \Rightarrow e_n \}.
\]

First, $e$ is evaluated to a value $v$; then the patterns are tried in order
until one matches. Pattern matching produces a \emph{binding environment}
$\theta$ (from pattern variables to subvalues), which is combined with
$\rho$ to evaluate the chosen branch.

We write:
\[
  \textsf{match}(p, v) = \theta
\]
to mean: pattern $p$ matches value $v$ and returns bindings $\theta$, or
is undefined if $p$ does not match $v$.

\paragraph{Pattern Matching Auxiliary Rules}

We define $\textsf{match}(p,v)$ by cases:

\[
\inference[\textsc{M-Wild}]{}{
  \textsf{match}(\_, v) = \emptyset
}
\]

\[
\inference[\textsc{M-Var}]{}{
  \textsf{match}(x, v) = [x \mapsto v]
}
\]

\[
\inference[\textsc{M-Const}]{}{
  c = v
}{
  \textsf{match}(c, v) = \emptyset
}
\]

\[
\inference[\textsc{M-Constr}]{
  v = C(v_1,\dots,v_k) \\
  \textsf{match}(p_1, v_1) = \theta_1 \quad \cdots \quad
  \textsf{match}(p_k, v_k) = \theta_k \\
  \text{dom}(\theta_i) \cap \text{dom}(\theta_j) = \emptyset \text{ for } i \neq j
}{
  \textsf{match}(C(p_1,\dots,p_k), C(v_1,\dots,v_k)) =
    \theta_1 \cup \cdots \cup \theta_k
}
\]

If no rule applies, $\textsf{match}(p,v)$ is undefined.

\paragraph{Evaluation of \texttt{match}}

We write the branches as a list:

\[
\mathcal{B} = (p_1 \Rightarrow e_1, \dots, p_n \Rightarrow e_n).
\]

We define:

\[
\inference[\textsc{E-Match}]{
  \rho \vdash e \Downarrow v \\
  \textsf{firstMatch}(\rho, v, \mathcal{B}) = (\theta, e_k) \\
  (\rho \cup \theta) \vdash e_k \Downarrow v'
}{
  \rho \vdash \texttt{match } e \{ \mathcal{B} \} \Downarrow v'
}
\]

where $\textsf{firstMatch}$ is a meta--level function that searches
the branch list from left to right:

\[
\textsf{firstMatch}(\rho, v, p_i \Rightarrow e_i, \dots) =
  \begin{cases}
    (\theta, e_i) & \text{if } \textsf{match}(p_i, v) = \theta, \\
    \textsf{firstMatch}(\rho, v, \text{rest}) & \text{otherwise}.
  \end{cases}
\]

If no branch matches, the semantics is undefined (a static exhaustiveness
check in Kleis prevents this in well-typed programs).

\subsection*{Determinism}

For this core fragment the semantics is deterministic:

\[
  \text{If } \rho \vdash e \Downarrow v_1 \text{ and } \rho \vdash e \Downarrow v_2,
  \text{ then } v_1 = v_2.
\]

A proof proceeds by induction on derivations of $\rho \vdash e \Downarrow v$.

\subsection*{Extensions}

Additional Kleis constructs (summations, integrals, derivatives, structure
operations, etc.) may be given semantics by extending:

\begin{itemize}
  \item the space of values (e.g.\ functionals, tensors, matrices),
  \item the primitive interpretation function $\llbracket \text{op} \rrbracket$,
  \item the environment to include structure instances and built-in operators.
\end{itemize}

These extensions are conservative over the rules given above.

\end{document}