\documentclass[12pt]{article}
\usepackage{amsmath,amssymb}
\usepackage{mathpartir}
\usepackage{geometry}
\geometry{margin=1in}

\begin{document}

\section*{Pattern-Matching Exhaustiveness Lemma}

We work with a fixed algebraic data type declaration

\[
  \texttt{data } D(\vec{\alpha}) = C_1(\vec{\tau}_1) \mid \cdots \mid C_n(\vec{\tau}_n).
\]

Values of type $D(\vec{\sigma})$ are of the form
\[
  C_i(v_1,\dots,v_k)
\]
with $v_j$ values and $C_i$ one of the constructors above.

\subsection*{Exhaustiveness Judgment}

We introduce a judgment
\[
  \Delta \vdash \mathsf{Exh}(D, \mathcal{P})
\]
read: \emph{under type context $\Delta$, the pattern multiset $\mathcal{P}$ is exhaustive for type $D$}.

We write a branch list as
\[
  \mathcal{P} = \{\,p_1, \dots, p_m\,\}.
\]

We define exhaustiveness inductively by the following inference rules.

\paragraph{Wildcard and Variable Patterns}

A single wildcard (or variable) pattern is exhaustive for any type:

\[
\inference[\textsc{Ex-Wild}]{}{
  \Delta \vdash \mathsf{Exh}(T,\{\_\})
}
\]

\[
\inference[\textsc{Ex-Var}]{}{
  \Delta \vdash \mathsf{Exh}(T,\{x\})
}
\]

\paragraph{Constructor Families}

Assume $D$ has constructors $C_1,\dots,C_n$ as above.
For each $i$, let $\vec{\tau}_i = (\tau_{i1},\dots,\tau_{ik_i})$ denote the argument types.

We consider pattern families of the form
\[
  \mathcal{P} = \{\,C_1(\vec{p}_1), \dots, C_n(\vec{p}_n)\,\}
\]
where $\vec{p}_i$ is a tuple $(p_{i1},\dots,p_{ik_i})$.

We require that for each argument position $j$ of each constructor $C_i$,
the collection of patterns at that position is itself exhaustive for the
corresponding argument type $\tau_{ij}$.

Formally, for each $i$ and $j$, let
\[
  \mathcal{P}_{i,j} = \{\,p_{ij} \mid C_i(\dots,p_{ij},\dots) \in \mathcal{P} \,\}.
\]

Then we define:

\[
\inference[\textsc{Ex-Constrs}]{
  \text{Constructors of $D$ are exactly } C_1,\dots,C_n
  \\[0.3em]
  \forall i,j.\ \Delta \vdash \mathsf{Exh}(\tau_{ij}, \mathcal{P}_{i,j})
}{
  \Delta \vdash \mathsf{Exh}(D,\{\,C_1(\vec{p}_1),\dots,C_n(\vec{p}_n)\,\})
}
\]

\paragraph{Adding Redundant Patterns}

If a set of patterns is exhaustive, adding more patterns preserves exhaustiveness:

\[
\inference[\textsc{Ex-Weakening}]{
  \Delta \vdash \mathsf{Exh}(T,\mathcal{P})
}{
  \Delta \vdash \mathsf{Exh}(T,\mathcal{P} \cup \{p\})
}
\]

(Here $\cup$ denotes multiset union.)

\subsection*{Typing Rule for \texttt{match}}

We extend the expression typing judgment with a rule for pattern matching:

\[
  \Gamma \vdash e : T \qquad
  \Gamma, \Gamma_{p_k} \vdash e_k : U \quad (1 \leq k \leq m) \qquad
  \Delta \vdash \mathsf{Exh}(T,\{p_1,\dots,p_m\})
\]

where $\Gamma_{p_k}$ is the context of bindings introduced by pattern $p_k$,
e.g.\ for $C(x,y)$ we have $\Gamma_{C(x,y)} = \{x:\tau_1, y:\tau_2\}$.

Then:

\[
\inference[\textsc{T-Match}]{
  \Gamma \vdash e : T
  \\
  \forall k.\ \Gamma, \Gamma_{p_k} \vdash e_k : U
  \\
  \Delta \vdash \mathsf{Exh}(T,\{p_1,\dots,p_m\})
}{
  \Gamma \vdash
    \texttt{match } e
      \{\,p_1 \Rightarrow e_1 \mid \dots \mid p_m \Rightarrow e_m\,\}
  : U
}
\]

The third premise enforces \emph{exhaustiveness} of the branch patterns.

\subsection*{Pattern-Matching Exhaustiveness Lemma}

We now state the central meta-theoretic property.

\paragraph{Lemma (Exhaustiveness).}

Let $\Gamma \vdash e : D(\vec{\sigma})$ and
\[
  \Gamma \vdash
    \texttt{match } e
      \{\,p_1 \Rightarrow e_1 \mid \dots \mid p_m \Rightarrow e_m\,\}
  : U
\]
be derived using rule \textsc{T-Match} above, so that in particular
\[
  \Delta \vdash \mathsf{Exh}(D(\vec{\sigma}),\{p_1,\dots,p_m\}).
\]

Assume a big-step evaluation semantics $\rho \vdash e \Downarrow v$ for values $v$
of type $D(\vec{\sigma})$.

Then for every environment $\rho$ consistent with $\Gamma$ and for every
$v$ with $\rho \vdash e \Downarrow v$, there exists an index $k$ and a
pattern environment $\theta$ such that:
\[
  \textsf{match}(p_k, v) = \theta
\]
is defined (i.e.\ branch $k$ matches), and hence the operational semantics
of the whole \texttt{match} expression is defined and cannot get stuck due
to an unmatched value.

\paragraph{Sketch of Proof.}

The proof proceeds by induction on the derivation of
\(
  \Delta \vdash \mathsf{Exh}(D,\{p_1,\dots,p_m\}).
\)

\begin{itemize}
  \item In cases \textsc{Ex-Wild} and \textsc{Ex-Var}, any value $v$ trivially
        matches the unique pattern.
  \item In case \textsc{Ex-Constrs}, any value of type $D$ is of the form
        $C_i(v_1,\dots,v_{k_i})$ for some $i$. The premise ensures that
        for each argument position $j$, the patterns $\mathcal{P}_{i,j}$
        are exhaustive for the corresponding type $\tau_{ij}$, so the
        induction hypothesis yields matching subpatterns for each argument,
        which combine to a matching constructor pattern.
  \item In case \textsc{Ex-Weakening}, adding extra patterns cannot
        destroy the existence of a matching branch.
\end{itemize}

Thus, well-typed \texttt{match} expressions with an \textsc{Exh}-justified
pattern set are \emph{exhaustive}: evaluation cannot get stuck due to an
unmatched constructor.

\end{document}