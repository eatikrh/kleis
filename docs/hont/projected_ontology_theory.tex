
\documentclass[12pt]{article}
\usepackage{amsmath}
\usepackage{amsfonts}
\usepackage{geometry}
\geometry{margin=1in}
\title{Projected Ontology Theory (POT): A Formal Statement}
\author{}
\date{}

\begin{document}

\maketitle

\section*{Core Hypothesis}

Projected Ontology Theory (POT) posits that the fundamental ontological layer of the universe is not four-dimensional spacetime $\mathbb{R}^4$, but a higher-order Hilbert space referred to as \textbf{Hont} (Hilbert Ontology). Observable physical phenomena in $\mathbb{R}^4$ are projections of structured modal flows within this ontological space.

\section*{Key Constructs}

\begin{itemize}
  \item \textbf{Hont:} A separable, infinite-dimensional Hilbert space $\mathcal{H}$ whose elements are modal functions, potentially band-limited and constrained by physical symmetries.
  \item \textbf{Modal Flow:} A continuous evolution $\phi(\tau)$ of elements in $\mathcal{H}$ governed by internal dynamics and subject to spectral constraints.
  \item \textbf{Projection:} A mapping $\Pi: \mathcal{H} \to \mathbb{R}^4$ which defines the appearance of modal structures as observable physical fields, particles, and interactions.
\end{itemize}

\section*{Derived Ontology}

\begin{itemize}
  \item \textbf{Mass:} Arises from the residue of poles in the Fourier spectrum of the modal flow $\phi$. It is ontological, not a property emergent from spacetime curvature.
  \item \textbf{Charge:} Emerges from phase winding numbers in modal functions, associated with topological features in Hont.
  \item \textbf{Time:} Is a projection of modal continuity; the apparent arrow of time arises from the directed evolution of modal sequences under projection.
  \item \textbf{Spacetime:} $\mathbb{R}^4$ is not fundamental but a structured image under $\Pi$ of underlying modal states.
\end{itemize}

\section*{Projection Principles}

\begin{itemize}
  \item Projection kernels resemble Green’s functions, convolving modal flow to generate observable dynamics.
  \item Projected observables (fields, particles) are filtered band-limited representations of modal bundles.
  \item The richness of physical diversity (e.g., stars, particles) is constrained by the bandwidth of modal flow.
\end{itemize}

\section*{Interpretive Commitments}

\begin{itemize}
  \item Ontological priority is given to modal structures in Hont, not to their projected appearances.
  \item Measurement and apparent locality in $\mathbb{R}^4$ are derivative phenomena.
  \item The Schrödinger equation and field theories in spacetime arise as effective projections of more fundamental modal dynamics.
\end{itemize}

\section*{Extensions}

\begin{itemize}
  \item Modal functions may be generalized from scalar-valued $\mathbb{C}$ to vector- or group-valued codomains $\mathbb{C}^n$, $SU(n)$, etc.
  \item Spin, flavor, and interaction structure emerge from internal symmetries in Hont.
  \item Projection geometry may yield nontrivial topology, leading to observable quantization and conservation laws.
\end{itemize}

\section*{Status}

Projected Ontology Theory (POT) offers a non-spacetime ontological basis for fundamental physics, where observable reality is understood as a shadow of a structured and evolving Hilbert space.

\end{document}
