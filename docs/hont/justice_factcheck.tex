
\documentclass{article}
\usepackage{geometry}
\geometry{margin=1in}
\title{Justice and Crime: A Structural Correction}
\author{}
\date{}

\begin{document}

\maketitle

\section*{Statement}
\textit{“We have centuries of writing about how to punish crime but very little about how to recognize justice as modal symmetry.”}

\section*{Fact Check: Confirmed}

\subsection*{1. Centuries of Writing on Crime and Punishment}
This part is historically accurate.

Human legal and philosophical traditions have long emphasized:
\begin{itemize}
  \item Defining criminal behavior (e.g., Hammurabi’s Code, Roman law, Sharia, Canon Law)
  \item Designing systems of punishment (e.g., retribution, deterrence, rehabilitation)
  \item Literary treatments of crime and punishment (e.g., Dostoevsky, Camus, Kafka)
\end{itemize}

These writings primarily focus on external regulation, codified norms, and institutional enforcement. Justice is often viewed through the lens of procedure or retribution.

\subsection*{2. Little Writing on Justice as Modal Symmetry}
This part is also accurate.

While classical philosophers (e.g., Plato, Aristotle) and modern theorists (e.g., Rawls, Nozick) have defined justice in various abstract ways, they rarely:
\begin{itemize}
  \item Frame justice as a structural invariant across modal domains,
  \item Treat justice as the preservation of phase or projection coherence,
  \item Ground justice ontologically rather than normatively.
\end{itemize}

Even metaphysical models such as Plato’s Forms do not explicitly formulate justice as a mode-preserving relation under projection.

\subsection*{Conclusion}
The statement is factually and philosophically accurate.

\textbf{Comparian Contribution:} Justice is not merely a social or legal construct. It is a modal symmetry—an ontological constraint on what counts as coherent projection. Crime and punishment are projections; justice precedes them.

\end{document}
