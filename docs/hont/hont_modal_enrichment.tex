
\documentclass[12pt]{article}
\usepackage{amsmath}
\usepackage{amsfonts}
\usepackage{geometry}
\geometry{margin=1in}
\title{Modal Flow Enrichment in Hont}
\author{}
\date{}

\begin{document}

\maketitle

\section*{Background}

In our scalar-valued formulation of modal flow in Hont, the modal function was defined as:
\[
\phi: X \times \mathbb{R}_\tau \to \mathbb{C}
\]
This structure allowed us to extract meaningful ontological quantities such as:
\begin{itemize}
  \item Mass (as the residue of a pole in modal frequency space)
  \item Electric charge (as phase winding number)
  \item Time (as continuity in modal flow $\tau$)
\end{itemize}

\section*{Enriched Model}

We now propose an enriched structure for the modal function:
\[
\phi: X \times \mathbb{R}_\tau \to \mathbb{C}^n
\]
where:
\begin{itemize}
  \item $n \in \mathbb{N}$ is the dimension of internal degrees of freedom (e.g., flavor, spin)
  \item Each component $\phi_i$ evolves under the same modal principles: band-limited, residue-bearing, phase-structured
\end{itemize}

\section*{Purpose}

This formulation aims to:
\begin{itemize}
  \item Account for observed phenomena such as neutrino flavor oscillations without introducing ``flavors of mass''
  \item Allow richer symmetry (e.g., $SU(n)$) while preserving:
  \begin{itemize}
    \item Ontological mass structure (via eigenvalues of a mass matrix $\mathcal{M}$)
    \item Phase-based charge conservation
    \item Projection into $\mathbb{R}^4$ via Green’s function convolution
  \end{itemize}
\end{itemize}

\section*{Preservation}

The $\mathbb{C}^n$-valued modal field retains:
\begin{itemize}
  \item Green’s function formulation with poles at physical masses
  \item Band limitation
  \item Scalar theory as a special case ($n = 1$)
  \item Topological interpretation of charge and mass
\end{itemize}

\section*{Next Steps}

\begin{enumerate}
  \item Explore the general structure of the mass operator $\mathcal{M}$
  \item Quantify how projection kernels map enriched modal flow into observable fields
  \item Extend this to include interactions and possibly spinor-valued fields
\end{enumerate}

\section*{Status}

Scalar-valued modal flow is not abandoned, but $\mathbb{C}^n$-valued flow is adopted as the \textbf{default formalism} for modeling systems with internal structure.

\end{document}
