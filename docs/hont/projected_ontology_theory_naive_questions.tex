
\documentclass{article}
\usepackage{amsmath, amssymb}
\usepackage{geometry}
\geometry{margin=1in}
\title{Projected Ontology Theory (POT)}
\author{}
\date{}

\begin{document}

\maketitle

\section*{Formal Definition Draft v0.1}

\subsection*{Premise}
Projected Ontology Theory (POT) posits that the ontologically real substrate of the universe is a \textbf{Hilbert space} \( \mathcal{H} \), and that our observed 4D spacetime \( \mathbb{R}^4 \) is a \textbf{projection} from this underlying structure. Observable physical phenomena (mass, charge, time, etc.) arise as \emph{stable eigenmodes} or \emph{projections} from \( \mathcal{H} \), governed by dynamics intrinsic to the Hilbert structure.

\subsection*{Core Postulates}
\begin{enumerate}
\item \textbf{Ontological Substrate} \\
Reality exists in a separable complex Hilbert space \( \mathcal{H} \) with an inner product structure and a complete basis of eigenstates.

\item \textbf{Projection to Observable Space} \\
The 4-dimensional physical spacetime \( \mathbb{R}^4 \) and all observables therein are derived via a projection operator \( \Pi: \mathcal{H} \rightarrow \mathbb{R}^4 \), possibly parameterized by eigenvalues of time and space operators.

\item \textbf{Modal Flow} \\
Dynamics in \( \mathbb{R}^4 \) are a consequence of cyclic or quasi-cyclic modal flows in \( \mathcal{H} \), possibly involving resonance phenomena and governed by functional equations or Green’s functions in \( \mathcal{H} \).

\item \textbf{Emergence of Mass and Charge} \\
Physical quantities like mass and charge correspond to residues or spectral densities derived from modal analysis in \( \mathcal{H} \), not intrinsic attributes of particles in \( \mathbb{R}^4 \).

\item \textbf{Collapse and Measurement} \\
Measurement is interpreted as a transfer of energy from \textbf{propagating wave modes} (traveling solutions) to \textbf{eigenmodes} (stationary, projectable modes) rather than a stochastic collapse.

\item \textbf{Time and Causality} \\
Time emerges from the structure of projection itself. Modal causality in \( \mathcal{H} \) can appear time-symmetric, but the projection imposes an arrow of time due to mode selectivity.
\end{enumerate}

\subsection*{Mathematical Sketch}
Let:
\begin{itemize}
\item \( \psi \in \mathcal{H} \)
\item \( \hat{O} \) be an observable (self-adjoint operator)
\item \( \Pi \) be a projection operator from \( \mathcal{H} \to \mathbb{R}^4 \)
\end{itemize}

Then:
\begin{itemize}
\item The observed value in \( \mathbb{R}^4 \) is \( \Pi(\hat{O} \psi) \)
\item The evolution is governed by modal interaction:
\[
i\hbar \frac{d\psi}{dt} = \hat{H} \psi
\]
but \( \hat{H} \) includes not just Hamiltonians on \( \mathbb{R}^4 \), but structural potentials of \( \mathcal{H} \) (e.g., resonance constraints, mode entanglements)
\end{itemize}

\subsection*{Forbidden but Fundamental Questions}
These are questions rarely asked in public lectures, papers, or classrooms—yet they strike at the core of our assumptions:
\begin{itemize}
\item How can the scalar-valued Schrödinger equation describe the structure and interaction of trillions of entities?
\item Why do we allow ourselves to define ``particles'' as objects in higher-order codomains—are these inventions or discoveries?
\item Can a single wavefunction meaningfully encode locality, identity, entanglement, and emergence?
\item Are so-called fundamental constants like \( \varepsilon_0 \), \( \mu_0 \), and even \( c \) truly fundamental—or projection redundancies?
\item Why are there three neutrino flavors? Why not one, or twenty-seven?
\item Does the Higgs mechanism really explain mass, or is it a formal trick to preserve gauge symmetry in a broken theory?
\item What is a neutrino ontologically? Is it an object, or a phase-preserving whisper of modal interference?
\end{itemize}

These questions may sound naive, but they expose critical blind spots in the architecture of modern physics—and provide the philosophical and formal foundation upon which POT is constructed.

\vspace{1em}
To be refined in version v0.2: treatment of locality, entanglement, observer dependence, and categorical formulation.

\end{document}
