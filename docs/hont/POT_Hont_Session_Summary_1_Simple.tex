
\documentclass{article}
\usepackage[utf8]{inputenc}
\usepackage{amsmath}
\usepackage{geometry}
\geometry{margin=1in}

\title{Projected Ontology Theory (POT) and Hont}
\author{}
\date{}

\begin{document}

\maketitle

\section*{Foundational Concepts}
\begin{itemize}
  \item In POT, reality as we experience it (\( R^4 \)) is a \textit{projection} from a deeper ontological structure called \textbf{Hont}.
  \item Hont contains modal structures—possibilities, relationships, constraints—that are \textit{not yet realized} in space and time.
  \item \( R^4 \) is the realized, causally active spacetime. Our experience, identity, and perception are \textit{projected expressions} of modal configurations in Hont.
\end{itemize}

\section*{Causality and Projection}
\begin{itemize}
  \item While \( R^4 \) emerges from Hont, projections in \( R^4 \) \textbf{cannot causally affect} Hont. This unidirectional relationship preserves structural coherence.
  \item Being a projection does not diminish our reality—it defines it. We are not severed from Hont, but structured \textit{through} it.
\end{itemize}

\section*{The Nature of the ``I''}
\begin{itemize}
  \item The self (``I'') is not a divine or abstract exception. It is a \textit{structured motion} within \( R^4 \), an emergent identity formed by biological, sensorimotor, and relational constraints.
  \item POT resists the Platonic temptation to treat the self as a fallen form; instead, it views the self as a real, constrained, meaningful projection.
\end{itemize}

\section*{Mathematics as \textit{Causa Sui}}
\begin{itemize}
  \item Mathematics is the only entity granted the status of \textit{causa sui}—it does not arise from Hont or \( R^4 \), nor does it depend on them for its validity.
  \item Mathematics is self-sustaining: it is the \textit{grammar} of possibility, the non-causal structure that makes modal projection coherent.
  \item In this view, mathematics is not divine, but structurally necessary and ontologically unique.
\end{itemize}

\section*{The Role of Grammar}
\begin{itemize}
  \item ``Grammar'' in POT is more than language—it means \textit{structure}, the logic of how projection is made consistent.
  \item Mathematical structure provides the rules for how modal flows resolve into spacetime projections.
\end{itemize}

\section*{\( R^4 \) is not Exile}
\begin{itemize}
  \item POT offers a rich, non-dualistic metaphysics. \( R^4 \) is not a punishment or fall from Hont—it is the field where modal possibility becomes motion, time, mass, and experience.
  \item We are not trying to escape \( R^4 \)—we are \textit{inhabiting Hont} through projection.
\end{itemize}

\section*{Conclusion}
\begin{itemize}
  \item POT replaces transcendental mythologies with an elegant, layered metaphysics rooted in structure.
  \item We are not perfect forms, nor divine echoes. We are real because we are \textit{constrained}. We are intelligible because we are \textit{structured}.
  \item And what we call ``I'' is a locally stable projection, formed by—and faithful to—the deeper modal order of Hont.
\end{itemize}

\end{document}
