% Basic Calculus Examples for Golden Tests
% These are standard notation examples from various style guides

% === DERIVATIVES ===

% Ordinary derivative (Leibniz notation)
\frac{dy}{dx}

% Partial derivative
\frac{\partial f}{\partial x}

% Second derivative
\frac{d^2 y}{dx^2}

% Second partial derivative
\frac{\partial^2 f}{\partial x^2}

% Mixed partial derivative
\frac{\partial^2 f}{\partial x \, \partial y}

% Derivative with respect to time
\frac{d\mathbf{r}}{dt}

% === INTEGRALS ===

% Indefinite integral
\int f(x) \, dx

% Definite integral
\int_{a}^{b} f(x) \, dx

% Multiple integrals
\iint_{D} f(x,y) \, dx \, dy

\iiint_{V} f(x,y,z) \, dx \, dy \, dz

% Contour integral
\oint_{C} f(z) \, dz

% Surface integral
\iint_{S} \mathbf{F} \cdot d\mathbf{S}

% === LIMITS ===

% Basic limit
\lim_{x \to a} f(x)

% Limit at infinity
\lim_{x \to \infty} \frac{1}{x} = 0

% Two-sided limit
\lim_{h \to 0} \frac{f(x+h) - f(x)}{h}

% === SUMS AND PRODUCTS ===

% Finite sum
\sum_{i=1}^{n} i = \frac{n(n+1)}{2}

% Infinite sum
\sum_{n=1}^{\infty} \frac{1}{n^2} = \frac{\pi^2}{6}

% Product
\prod_{i=1}^{n} i = n!

% === SERIES ===

% Taylor series
f(x) = \sum_{n=0}^{\infty} \frac{f^{(n)}(a)}{n!} (x-a)^n

% Power series
\sum_{n=0}^{\infty} c_n (x-a)^n

